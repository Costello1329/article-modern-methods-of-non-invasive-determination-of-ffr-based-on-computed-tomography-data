\input {preamble}\begin {document}\fancyhead {}\fancyfoot {}
\par \noindent \textbf {\textnumero 1}\par \begin {otherlanguage*}{english}\makeatletter \textbf {Leladze \,Konstantin\unhbox \voidb@x \nobreak \,G.\unskip {}, \ignorespaces Alexandrova \,Larisa\unhbox \voidb@x \nobreak \,V.}\makeatother \par \textit {Modern Methods of Non-invasive Determination of FFR Based on Computed Tomography Data}\par This article provides a review of contemporary methods for constructing three-dimensional models of internal organs and blood vessels based on computed tomography (CT) imaging data in the context of non-invasive determination of fractional flow reserve (FFR). The study covers various algorithmic approaches aimed at improving the accuracy and reliability of reconstructing the surface geometry of internal organs and vessels. Key modern approaches, including segmentation algorithms, boundary detection methods, and machine learning techniques in image processing, are discussed. \par \keywordsname : Fractional Flow Reserve (FFR), non-invasive imaging, computed tomography (CT), 3D reconstruction, image segmentation, coronary artery disease (CAD), machine learning \end {otherlanguage*}\par \par \medskip 
\par \begin {otherlanguage*}{russian}\makeatletter \textbf {Леладзе ~К.\unhbox \voidb@x \nobreak \,Г.\unskip {}, \ignorespaces Александрова ~Л.\unhbox \voidb@x \nobreak \,В.}\makeatother \par \textit {Современные методы неинвазивного определения ФРК на основании данных компьютерной томографии}\par В данной статье представлен обзор современных методов построения трёхмерных моделей внутренних органов и сосудов на основании данных компьютерной томографии (КТ) в контексте задачи неинвазивного определения фракционного резервного кровотока (ФРК). Исследование охватывает различные алгоритмические подходы, применяемые для повышения точности и надёжности реконструкции геометрии поверхности внутренних органов и сосудов. Рассмотрены ключевые современные подходы, включая алгоритмы сегментации, методы на основании детектирования границ и методы машинного обучения в обработке изображений. \par \keywordsname : Фракционный резерв кровотока (ФРК), неинвазивная визуализация, компьютерная томография (КТ), трёхмерная реконструкция, сегментация изображений, ишемическая болезнь сердца (ИБС), машинное обучение \par \end {otherlanguage*}\par \vspace \bigskipamount 
\par \begin {otherlanguage*}{english}\makeatletter \textbf {Леладзе ~К.\unhbox \voidb@x \nobreak \,Г.\unskip {}, \ignorespaces Александрова ~Л.\unhbox \voidb@x \nobreak \,В.}\makeatother \par \textit {Современные методы неинвазивного определения ФРК на основании данных компьютерной томографии}\par В данной статье представлен обзор современных методов построения трёхмерных моделей внутренних органов и сосудов на основании данных компьютерной томографии (КТ) в контексте задачи неинвазивного определения фракционного резервного кровотока (ФРК). Исследование охватывает различные алгоритмические подходы, применяемые для повышения точности и надёжности реконструкции геометрии поверхности внутренних органов и сосудов. Рассмотрены ключевые современные подходы, включая алгоритмы сегментации, методы на основании детектирования границ и методы машинного обучения в обработке изображений. \par \keywordsname : Фракционный резерв кровотока (ФРК), неинвазивная визуализация, компьютерная томография (КТ), трёхмерная реконструкция, сегментация изображений, ишемическая болезнь сердца (ИБС), машинное обучение \par \end {otherlanguage*}\par \vspace \bigskipamount 
